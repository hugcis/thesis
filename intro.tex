\chapter{Introduction}

% Intro should be readable to a non expert

\section{Goal}

The goal of this thesis is to develop methods for studying and using the
computations of complex dynamical systems to construct learning algorithms that
use a limited amount of supervision. We break down this objective into the
following subgoals: (i) Identify complex dynamical systems with the potential to
exhibit emergent open-ended growth of complexity and evolutionary-like
properties. (ii) Measure the fraction of systems that have the most complex and
rapidly evolving behavior. Defining these notions is also part of the goal. We
believe those systems to be the most promising for further use. (iii) Work on
applying these promising systems to challenging tasks where classical machine
learning models might fail or may prove less efficient.

This work is not about

In this thesis we work towards attaining these goal, focusing on one particular
complex system: the \acl{CA}. This model has been studied extensively
because of its simple definition yet surprisingly complex behavior. We provide a
more precise definition of \aclp{CA} in Section .


\section{Motivation}

Most of existing machine learning algorithms rely on the choice of an objective,
function: a clearly defined mapping from the current state and parameters of a
model to a real value, which indicates the performance of that model. The
function depends on the goal of the model. For a supervised learning problem, we
may count the number of misclassified objects or a distance between the
predictions and expected results. In unsupervised learning, the family of
algorithms learning from untagged data, objectives are still used. For instance,
the well-known K-means clustering algorithm minimizes the sum of square distance
to cluster centers.

This reliance on objective functions creates two main issues: (i) the objective
is not always clearly defined. For example, the objective function of a walking
robot could be to ``not fall when stepping through its surrounding
environment''. But (ii) Using such functions as goals can be counter-productive,
because, as many example in nature demonstrate, the path to these objectives are
often deceptive. They involve going in a unexpected direction that may initially
seem against the original goal \cite{stanleyWhyGreatnessCannot2015}.

In this thesis, the term \emph{unsupervised} refers to a form of learning with
no predefined objective. Like for natural evolution, we expect unsupervised
algorithms to develop new features autonomously and become progressively more
complex over time. Such algorithms would regularly learn to solve problems on
their own without the need to explicitly guide them, thereby discovering robust
and diverse solutions to deceptive problems.


\section{Challenges}

The study of complex systems poses a range of challenges in and of itself
\cite{sanmiguelChallengesComplexSystems2012}. The complexity of a system is an
emergent property that arises from its intricate structure, its number of
elements, how it functions, and how it responds to different kinds of external
influence.

\section{Contributions}

The main contributions of this thesis are
\begin{enumerate}
  \item A thorough literature review making the connection between cellular
        automata and other complex systems literature and machine learning
        literature.

  \item The development of a general complexity metric that can help identify
  complex systems with interesting behavior.

  \item The development of a coarse graining method for visualizing computations
        in cellular automata and other discrete systems with local interactions.

  \item The introduction of a metric for learning efficiency for learning
        algorithms as well as a benchmark dataset of progressively harder
        language tasks.
\end{enumerate}

\section{Thesis overview}

\section{Publications and software}
