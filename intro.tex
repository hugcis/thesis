\chapter{Introduction}

% Intro should be readable to a non expert

\section{Goal}

The goal of this thesis is to develop methods for studying and using the
computations of complex dynamical systems to construct learning algorithms that
use a limited amount of supervision.
We break down this objective into the following subgoals: (i)


\section{Motivation}

Most of existing machine learning algorithms rely on the choice of an objective,
function: a clearly defined mapping from the current state and parameters of a
model to a real value, which indicates the performance of that model. The
function depends on the goal of the model. For a supervised learning problem, we
may count the number of misclassified objects or a distance between the
predictions and expected results. In unsupervised learning, the family of
algorithms learning from untagged data, objectives are still used. For instance,
the well-known K-means clustering algorithm minimizes the sum of square distance
to cluster centers.

This reliance on objective functions creates two main
issues: (i) the objective is not always clearly defined. For example, the
objective function of a walking robot could be to ``not fall when stepping
through its surrounding environment''. But (ii) Using such functions as goals
can be counter-productive, because, as many example in nature demonstrate, the
path to these objectives are often deceptive. They involve going in a unexpected
direction that may initially seem against the original goal
\cite{stanleyWhyGreatnessCannot2015}.




\section{Challenges}

The study of complex systems poses a range of challenges

\section{Contributions}

The main contributions of this thesis are
\begin{enumerate}
  \item A thorough literature review making the connection between cellular
        automata and other complex systems literature and machine learning
        literature.

  \item The development of a general complexity metric that can help identify
  complex systems with interesting behavior.

  \item The development of a coarse graining method for visualizing computations
        in cellular automata and other discrete systems with local interactions.

  \item The introduction of a metric for learning efficiency for learning
        algorithms as well as a benchmark dataset of progressively harder
        language tasks.
\end{enumerate}

\section{Thesis overview}

\section{Publications and software}
