\chapter{Literature Review}
\label{cha:literature-review}

\section{Measuring complexity}


\section{Computing with complex systems}

The problem of computing within complex systems is closely related to the
question of decentralized parallel computation in general, for which there exist
an abundant literature.

\subsection{Computing in cellular automata}

\Acfp{CA} are decentralized parallel systems with many identical components with
local connectivity. Because of these properties, they have the potential to
carry out robust, efficient computations. \ac{CA}-based computing machines could
recover from perturbations or carry out computations in stochastic environments.
Moreover, they are also interesting for modeling the behavior of natural complex
systems. For more details about \acp{CA}, see section
\ref{sec:cellular-automata-sec}.

\paragraph{Evolving \acp{CA} with genetic algorithms}
A major next step from the clever but difficult hand-design of \acp{CA} rules to
perform computations is the use of learning algorithms to automatically design
those rules \parencite{mitchellEvolvingCellularAutomata1996}.

Genetic algorithms are a type of search method inspired by biological evolution
\parencite{bookerClassifierSystemsGenetic1989}. Candidate solutions to a problem
are encoded as \emph{chromosomes}, that is a structured data representation that
can be modified incrementally. In practice it is often chosen to be a string of
bits.

\paragraph{Reliable computation in cellular automata}
The problem of carrying out reliable computations in \acp{CA} is studied by
\cite{gacsReliableComputationCellular1986}

\subsection{Amorphous computing}
