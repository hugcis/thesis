\chapter{Literature Review}
\label{cha:literature-review}

\section{Measuring complexity}

\section{Emergence}

\section{Evolutionary algorithms}
Evolutionary algorithms are the class of algorithms inspired by natural
biological processes, in particular evolution.

\subsection{Novelty search}
The idea behind novelty search is to drive a search algorithm by the novelty of
produced behavior \parencite{lehmanAbandoningObjectivesEvolution2011}. The most
surprising feature of this algorithm is that the actual objective of the task is
not taken into account at all during the search process. Generated agents are
evaluated solely on the novelty of their behavior. Despite this, it appears to
be at least as efficient as other search processes that are goal focused, such
as maze navigation and biped locomotion
\parencite{lehmanAbandoningObjectivesEvolution2011}, swarm robotics
\parencite{gomesEvolutionSwarmRobotics2013}, or neural network design
\parencite{risiEvolvingPlasticNeural2010}.

\section{Open-ended evolution}
The field of Artificial Life research has worked to figure out what the
fundamental conditions for the emergence of living systems are, and how to
create a process that can display analogous levels of creativity and complexity
as natural evolution \parencite{eigenHypercycle1979,
  langtonArtificialLifeProceedings1989, dysonOriginsLife1999,
  stanleyWhyOpenEndednessMatters2019, packardOverviewOpenEndedEvolution2019,
  sorosOpenendednessLastGrand2017}. It build upon the data and understanding we
collected about the process of life, but abstracts from any specific living
process and attempts to integrate various approaches into one unified research
to extract the first principles of life. A major assumption underpinning this
research is that this natural evolution as a process can be implemented equally
well in different media \parencite{dennettDarwinDangerousIdea1996}.

A system that behaves like natural evolution, producing a seemingly endless
amount of novelty and complexity starting from elementary building blocks is
called \emph{open-ended}. The main challenge of \ac{OEE} is there is not a simple
single test for the phenomenon, but instead there are different kinds of
open-ended evolution. Systems can exhibit more than one kind at a time. In the
report from a workshop on \ac{OEE} at York
\parencite{taylorOpenEndedEvolutionPerspectives2016}, the authors summarized the
different kinds of \ac{OEE}, which were further refined in a follow up work
\parencite{packardOverviewOpenEndedEvolution2019}. They are as follows

\begin{enumerate}
  \item Interesting new kinds of entities and interactions
  \item Evolution of evolvability
  \item Major transitions
  \item Semantic evolution
\end{enumerate}

The first category describes the ability of a system to construct new entities
with different properties, behavior, or interactions with other entities. For
example, the Tierra simulation sees entities emerge that exploit the computing
power of others, acting as parasites. The second type is related to

\subsection{Defining open-endedness}

Several necessary conditions and requirements have been identified across the
\ac{OEE} literature, which form an overlapping set of potential research
directions for developing open-ended systems. We list a few of these conditions
here. First, \cite{maleyFourStepsOpenended1999} identifies four requirements:

\begin{enumerate}
  \item An open-ended evolutionary system must demonstrate unbounded diversity
        during its growth phase.
  \item An open-ended evolutionary system must embody selection.
  \item An open-ended evolutionary system must exhibit continuing new adaptive
        activity.
  \item An open-ended evolutionary system must have an endogenous implementation
        of niches.
\end{enumerate}
Next, \cite{sorosIdentifyingNecessaryConditions2014} identified four more necessary
conditions which are as follows

\begin{enumerate}
  \item A rule should be enforced that individuals must meet some minimal
        criterion (MC) before they can reproduce, and that criterion must be
        nontrivial.
  \item The evolution of new individuals should create novel opportunities for
        satisfying the MC.
  \item Decisions about how and where individuals interact with the world should
        be made by the individuals themselves.
  \item The potential size and complexity of the individuals' phenotypes should
        be (in principle) unbounded.
\end{enumerate}
Then, \cite{taylorRequirementsOpenEndedEvolution2015} also stated five
requirements for an open-ended system:

\begin{enumerate}
  \item Robustly reproductive individuals.
  \item A medium allowing the possible existence of a practically unlimited
        diversity of individuals and interactions, at various levels of
        complexity.
  \item Individuals capable of producing more complex offspring.
  \item An evolutionary search space that typically offers mutational pathways
        from one viable individual to other viable (and potentially fitter )
        individuals.
  \item Drive for continued evolution.
\end{enumerate}
Taylor also states a more general condition that should be sufficient for
creating an open-ended system as ``evolutionary dynamics in which new,
surprising, and sometimes more complex organisms continue to appear''
\parencite{taylorRequirementsOpenEndedEvolution2015,
  taylorOpenEndedEvolutionPerspectives2016}.

All these conditions illustrate a major challenge of \ac{OEE} research: the lack
of clear definition or notion of what conditions make a system open-ended. We
seem to agree about what is not open-ended, but whenever a constraint or
requirement for \ac{OEE} is identified, subsequent evidence forces us to refine
them later. This is related to the problem of complexity, for which no single
definition exists \parencite{johnsonSimplyComplexityClear2009}.

A common process to produce systems that behave analogously to natural evolution
is to start from evolvable units or building blocks
\parencite{srayApproachSynthesisLife1991, simsEvolvingVirtualCreatures1994,
  ofriaAvidaSoftwarePlatform2004, yaegerComputationalGeneticsPhysiology1994,
  channonImprovingStillPassing2003, spectorDivisionBlocksOpenended2007,
  sorosIdentifyingNecessaryConditions2014}. The reason is that starting from
higher level primitive units whose emergence would be hard to characterize may
be easier and faster than starting from lower level components. Results obtained
from these systems are often surprising, since they bear some key resemblance to
natural evolutionary processes. For example we note the emergence of parasitic
entities within the Tierra simulation \parencite{srayApproachSynthesisLife1991}.
The process of emergence of these building blocks from simple rules and
components has also been investigated significantly
\parencite{bagleySpontaneousEmergenceMetabolism1991,
  huttonEvolvableSelfReproducingCells2007, flammEvolutionMetabolicNetworks2010,
  sayamaSeekingOpenendedEvolution2011}. It appears harder to bridge the gap and
create high level evolutionary-like processes and behaviors from elementary
rules and substrates.

\subsection{Open-endedness in cellular automata}
One class of system that has rich interactions between each of its components as
well as no predefined evolvable units or assumptions about individuality is the
\ac{CA}. One of the very first automata, Von Neumann's self-reproducing machine,
was designed with goals that align with open-ended evolution, which is to build
a machine with no central controller and limited local interaction that can
self-reproduce as a whole
\parencite{vonneumannTheorySelfreproducingAutomata1966,
  pesaventoImplementationNeumannSelfReproducing1995}. Later, other
self-replicating structures such as Langton's loop
\parencite{langtonSelfreproductionCellularAutomata1984} and evoloops
\parencite{sayamaNewStructurallyDissolvable1999,
  salzbergComplexGeneticEvolution2004} showed that more properties of open-ended
systems can be included in a \ac{CA}. An potential limitation of \acp{CA} is the
absence of notion of conservation of ``matter''. For example, the game of life
can start from a configuration with very few live cells and create many more at
no cost during its evolution. Some authors believe that this conservation
property is essential to the construction of an open-ended evolving system
\parencite{taylorChapterCreativityEvolution2002}.

\subsection{Artificial chemistries}
Most of the time, \acp{AC} are not trying to accurately model chemical processes
(as in \parencite{ostrovskyCellularAutomataPolymer2001} for example). Instead,
they build models of the dynamics of complex molecular processes that lead to
evolutionary behavior \parencite{dittrichArtificialChemistriesReview2001}. By
abstracting away from the natural molecular processes, \acp{AC} tries to uncover
fundamental conditions for the emergence of organization, self-maintenance, or
self-construction with basic building blocks. There are various approaches for
building these \aclp{AC}, of which we review a few here.

\paragraph{Rewriting systems.}
Rewriting systems are composed of entities or symbols that get modified
according to a set of syntactic rules. Patterns of symbols or entities are
replaced according to these rules.

\paragraph{Cellular automata.}
\Acfp{CA} can be see as a particular case of lattice molecular systems. For
example the autopoietic system introduced by
\cite{varelaAutopoiesisOrganizationLiving1991} is a 2d square grid with sites
that be occupied by \emph{atoms} which is similar to a \ac{CA} with 4 states.
Each of these states is analogous to a chemical component of the system: ($\emptyset$)
empty site, (S) substrate, (K) catalyst and (L) monomer. Basic rules are applied
asynchronously and define how neighboring atoms interact with each other.
Remarkably, stable self-repairing cells arise spontaneously from these basic
rules. Their membrane is composed of a chain of monomers, which is maintained by
the substrate and catalyst reacting according to the rules. Some key components
were investigated in other works, showing that they are crucial for autopoeisis
to be possible \parencite{zelenySelforganizationLivingSystems1977,
  mcmullinRediscoveringComputationalAutopoiesis1997}.

\section{Computing with complex systems}

The problem of computing within complex systems is closely related to the
question of decentralized parallel computation in general, for which there exist
an abundant literature. Different paradigms exist for controlling and harvesting
the computations within complex systems. Several other names have been used for
closely related topics, such as organic computing
\parencite{muller-schloerOrganicComputingParadigm2011} which is the study of
systems with life-like properties such as self-organization or the ability to
adapt to a dynamically changing environment. Agent-based computing
\parencite{jenningsAgentBasedComputingPromise1999} focuses on computing systems
composed of several autonomous agents. Amorphous computing
\parencite{abelsonAmorphousComputing2000,
  nagpalProgrammablePatternFormationScaleIndependence2008,
  nagpalProgrammableSelfassemblyUsing2002} is about making vast quantities of
individual computing elements work, and to ensure ``the cooperation of vast
numbers of unreliable parts interconnected in unknown, irregular, and
time-varying ways''.

\subsection{Computing in cellular automata}

\Acfp{CA} are decentralized parallel systems with many identical components with
local connectivity. Because of these properties, they have the potential to
carry out robust, efficient computations. \ac{CA}-based computing machines could
recover from perturbations or carry out computations in stochastic environments.
Moreover, they are also interesting for modeling the behavior of natural complex
systems. For more details about \acp{CA}, see section
\ref{sec:cellular-automata-sec}.

\paragraph{Evolving \acp{CA} with genetic algorithms}
A major step forward from the clever but difficult hand-design of \acp{CA} rules
to perform computations is the use of learning algorithms to automatically
design those rules \parencite{mitchellEvolvingCellularAutomata1996}.

Genetic algorithms are a type of search method inspired by biological evolution
\parencite{bookerClassifierSystemsGenetic1989}. Candidate solutions to a problem
are encoded as \emph{chromosomes}, that is a structured data representation that
can be modified incrementally. In practice it is often chosen to be a string of
bits.

\paragraph{Reliable computation in cellular automata}
The problem of carrying out reliable computations in \acp{CA} is studied by
\parencite{gacsReliableComputationCellular1986}

\paragraph{Neural cellular automata}
\parencite{mordvintsevGrowingNeuralCellular2020,
  randazzoSelfclassifyingMNISTDigits2020, niklassonSelfOrganisingTextures2021}

\subsection{Amorphous computing}
